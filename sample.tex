\documentclass[openning]{ecustthsis}

% 设定文献翻译相关信息
\author{作者姓名}
% \class{作者班级}
% \studentNo{作者学号}
\title{华东理工大学本科生毕业论文模版}


% 可以按照个人的喜好自行设置字体
% 自行设置字体可参考此处
% \setmainfont{Times New Roman}
% \setCJKmainfont{SimSun}
% \setCJKfamilyfont{hei}{SimHei}
% 请注意:很多中文字体没有粗体的概念,因此在TeX中简单地使用\textbf是
% 不会出现粗体的(摊手),最好使用有独立的粗体和斜体的中文字体。如果使
% 用的字体粗体和斜体是单独的名称,请参考下面格式进行引用。

\begin{document}

% 模版使用形式和普通article类一样
% - \maketitle        显示标题;
% - \section{}        开始一个章节;
% - \subsection{}     一个次级章节;
% - \subsubsection{}  一个次次级章节;

\maketitle

% - 使用 \begin{abstract} 来开始和结束中文摘要。模版会自动加载关键词。

\ECUSTabstract
这是一份用于华东理工大学毕业环节的论文模版,主要包括了文献翻译、开题报告和论文正文的模版。

\ECUSTkeywords
华东理工,毕业论文, 本科生, \LaTeX


\section{导论}

传说中神一样编辑器的Emacs向来以难学难用,喜欢折腾人著称。三天打渔两天晒
网的我,居然心甘情愿地被它折腾了5,6年之久,期间苦乐不足为外人道也。

不过,以我的使用感觉,Emacs 更象是匹烈马:初时很难驾驭,可一旦征服,使
用起来便得心应手,威力无穷。尽管被它折腾的不轻,但也因此学会了很多提高
工作效率的小技巧。而在用 Emacs 编辑时更是可以做到心无旁骛,任由思路驰骋
纵横在键盘间,达到一种所谓“流”的状态。

虽说如此,长久以来,还是有很多小细节让自己在使用Emacs的时候很是不爽,最
近一周稍有闲暇,本着磨刀不误砍柴工的精神,也来折腾了一下 Emacs,居然被
我搞定了几个困扰已久的配置。整理记录一下,希望能帮到遇到同样问题的朋友
们。
\end{document}
